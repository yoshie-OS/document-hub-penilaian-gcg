\documentclass[12pt,a4paper]{article}
\usepackage[utf8]{inputenc}
\usepackage[T1]{fontenc}
\usepackage[english,indonesian]{babel}
\usepackage{geometry}
\usepackage{graphicx}
\usepackage{hyperref}
\usepackage{enumitem}
\usepackage{fancyhdr}
\usepackage{titlesec}
\usepackage{color}
\usepackage{float}

% Page setup
\geometry{left=2.5cm,right=2.5cm,top=2.5cm,bottom=2.5cm}

% Hyperlink setup
\hypersetup{colorlinks=true,linkcolor=blue,filecolor=magenta,urlcolor=blue,citecolor=blue}

% Header and footer setup
\pagestyle{fancy}
\fancyhf{}
\fancyhead[L]{\textbf{Manual Book Aplikasi GCG}}
\fancyhead[R]{\thepage}
\renewcommand{\headrulewidth}{0.4pt}

% Title formatting
\titleformat{\section}{\Large\bfseries\color{blue}}{\thesection}{1em}{}
\titleformat{\subsection}{\large\bfseries\color{darkgray}}{\thesubsection}{1em}{}
\titleformat{\subsubsection}{\normalsize\bfseries}{\thesubsubsection}{1em}{}

% Custom colors
\definecolor{lightblue}{RGB}{173,216,230}
\definecolor{darkblue}{RGB}{0,0,139}
\definecolor{darkgray}{RGB}{64,64,64}

\begin{document}

% Title page
\begin{titlepage}
\centering
\vspace*{2cm}
{\Huge\bfseries\color{darkblue} MANUAL BOOK APLIKASI\\[0.5cm] GOOD CORPORATE GOVERNANCE (GCG)\\[0.5cm] DOCUMENTS MANAGEMENT SYSTEM}
\vspace{2cm}
\includegraphics[width=0.3\textwidth]{placeholder_logo.png}
\vspace{2cm}
{\Large\bfseries Panduan Penggunaan Lengkap}
\vspace{1cm}
{\large Untuk Super Admin, Admin, dan User}
\vfill
{\large \textbf{Dibuat oleh:} Tim Pengembangan Aplikasi GCG}
\vspace{0.5cm}
{\large \textbf{Versi:} 1.0}
\vspace{0.5cm}
{\large \textbf{Tanggal:} Desember 2024}
\vspace{0.5cm}
{\large \textbf{Status:} Final}
\vspace{1cm}
\end{titlepage}

% Table of contents
\tableofcontents
\newpage

% Introduction
\section{Pendahuluan}
\subsection{Deskripsi Aplikasi}
Aplikasi Good Corporate Governance (GCG) Documents Management System adalah sistem manajemen dokumen yang dirancang untuk mengelola dokumen-dokumen terkait tata kelola perusahaan yang baik.

\subsection{Tujuan Manual}
Manual ini dibuat untuk membantu pengguna memahami cara menggunakan aplikasi GCG secara efektif dan efisien.

% Application Structure
\section{Struktur Aplikasi}
\subsection{Hierarki Pengguna}
\begin{verbatim}
Super Admin (Level Tertinggi)
├── Pengaturan Baru
├── Dashboard
├── Monitoring & Upload GCG
├── Arsip Dokumen
├── Performa GCG
└── AOI Management

Admin (Level Menengah)
└── Dashboard

User (Level Dasar)
└── Dashboard
\end{verbatim}

% Login and Authentication
\section{Login dan Autentikasi}
\subsection{Langkah Login}
\begin{enumerate}
\item \textbf{Buka Browser} dan akses URL aplikasi
\item \textbf{Halaman Login} akan muncul
\item \textbf{Masukkan kredensial}: Email/Username dan Password
\item \textbf{Klik tombol "Login"}
\item \textbf{Sistem akan memverifikasi} dan mengarahkan ke dashboard sesuai role
\end{enumerate}

\subsection{Placeholder Gambar}
\begin{figure}[H]
\centering
\includegraphics[width=0.8\textwidth]{placeholder_login.png}
\caption{Halaman Login - Form login dengan field email dan password}
\label{fig:login}
\end{figure}

% Super Admin Menu
\section{Menu Super Admin}
\subsection{Pengaturan Baru}
\subsubsection{Deskripsi}
Menu ini berfungsi untuk mengatur konfigurasi dasar sistem, termasuk manajemen tahun buku, struktur perusahaan, dan checklist GCG.

\subsubsection{Fitur Utama}
\begin{itemize}
\item \textbf{Manajemen Tahun Buku}: Menambah, mengedit, dan menghapus tahun buku
\item \textbf{Struktur Perusahaan}: Mengatur hierarki direktorat, subdirektorat, dan divisi
\item \textbf{Checklist GCG}: Membuat dan mengatur checklist dokumen yang diperlukan
\item \textbf{Manajemen Aspek}: Mengatur aspek-aspek GCG yang akan dimonitor
\end{itemize}

\subsection{Dashboard}
\subsubsection{Deskripsi}
Dashboard Super Admin menampilkan overview statistik dan performa sistem secara keseluruhan.

\subsection{Monitoring \& Upload GCG}
\subsubsection{Deskripsi}
Menu ini berfungsi untuk memonitor status upload dokumen GCG dari berbagai subdirektorat dan melakukan upload dokumen.

\subsection{Arsip Dokumen}
\subsubsection{Deskripsi}
Menu ini berfungsi untuk mengarsipkan dan mengelola dokumen GCG yang telah diupload.

\subsection{Performa GCG}
\subsubsection{Deskripsi}
Menu ini menampilkan analisis performa GCG dalam bentuk grafik dan statistik.

\subsection{AOI Management}
\subsubsection{Deskripsi}
Menu ini berfungsi untuk mengelola Area of Improvement (AOI) dan rekomendasi perbaikan.

% Admin Menu
\section{Menu Admin}
\subsection{Dashboard Admin}
\subsubsection{Deskripsi}
Dashboard Admin menampilkan informasi dan statistik dokumen sesuai dengan subdirektorat yang ditugaskan.

% General Features
\section{Fitur Umum}
\subsection{Manajemen Profil}
\subsection{Logout}
\subsection{Responsive Design}

% Troubleshooting
\section{Troubleshooting}
\subsection{Masalah Umum dan Solusi}
\subsubsection{Tidak Bisa Login}
\textbf{Gejala}: Error saat memasukkan username/password\\
\textbf{Solusi}: Pastikan username dan password benar, periksa koneksi internet, clear cache browser.

\subsubsection{File Upload Gagal}
\textbf{Gejala}: File tidak bisa diupload\\
\textbf{Solusi}: Periksa ukuran file (maksimal 10MB), pastikan format file didukung.

% Conclusion
\section{Kesimpulan}
Manual book ini telah menjelaskan secara lengkap cara menggunakan aplikasi GCG Documents Management System.

\subsection{Tips Penggunaan}
\begin{itemize}
\item \textbf{Baca manual} sebelum menggunakan aplikasi
\item \textbf{Simpan dokumen} secara teratur
\item \textbf{Backup data} penting secara berkala
\item \textbf{Update password} secara berkala
\item \textbf{Logout} setelah selesai menggunakan aplikasi
\end{itemize}

\end{document}

