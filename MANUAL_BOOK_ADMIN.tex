\documentclass[12pt,a4paper]{article}
\usepackage[utf8]{inputenc}
\usepackage[T1]{fontenc}
\usepackage[english,indonesian]{babel}
\usepackage{geometry}
\usepackage{graphicx}
\usepackage{hyperref}
\usepackage{enumitem}
\usepackage{fancyhdr}
\usepackage{titlesec}
\usepackage{xcolor}
\usepackage{float}

% Page setup
\geometry{left=2.5cm,right=2.5cm,top=2.5cm,bottom=2.5cm}

% Hyperlink setup
\hypersetup{colorlinks=true,linkcolor=blue,filecolor=magenta,urlcolor=blue,citecolor=blue}

% Header and footer setup
\pagestyle{fancy}
\fancyhf{}
\fancyhead[L]{\textbf{Manual Book Admin - Aplikasi GCG}}
\fancyhead[R]{\thepage}
\renewcommand{\headrulewidth}{0.4pt}

% Title formatting
\titleformat{\section}{\Large\bfseries\color{blue}}{\thesection}{1em}{}
\titleformat{\subsection}{\large\bfseries\color{darkgray}}{\thesubsection}{1em}{}
\titleformat{\subsubsection}{\normalsize\bfseries}{\thesubsubsection}{1em}{}

% Custom colors
\definecolor{lightblue}{RGB}{173,216,230}
\definecolor{darkblue}{RGB}{0,0,139}
\definecolor{darkgray}{RGB}{64,64,64}

\begin{document}

% Title page
\begin{titlepage}
\centering
\vspace*{2cm}
{\Huge\bfseries\color{darkblue} MANUAL BOOK ADMIN\\[0.5cm] GOOD CORPORATE GOVERNANCE (GCG)\\[0.5cm] DOCUMENTS MANAGEMENT SYSTEM}
\vspace{2cm}
\includegraphics[width=0.3\textwidth]{placeholder_logo.png}
\vspace{2cm}
{\Large\bfseries Panduan Penggunaan untuk Admin}
\vspace{1cm}
{\large Mengelola Dokumen GCG Subdirektorat}
\vfill
{\large \textbf{Dibuat oleh:} Tim Pengembangan Aplikasi GCG}
\vspace{0.5cm}
{\large \textbf{Versi:} 1.0}
\vspace{0.5cm}
{\large \textbf{Tanggal:} Desember 2024}
\vspace{0.5cm}
{\large \textbf{Status:} Final}
\vspace{1cm}
\end{titlepage}

% Table of contents
\tableofcontents
\newpage

% Introduction
\section{Pendahuluan}

\subsection{Deskripsi Aplikasi}
Aplikasi Good Corporate Governance (GCG) Documents Management System adalah sistem manajemen dokumen yang dirancang untuk mengelola dokumen-dokumen terkait tata kelola perusahaan yang baik. Sebagai Admin, Anda bertanggung jawab untuk mengelola dokumen GCG sesuai dengan subdirektorat yang ditugaskan.

\subsection{Tujuan Manual}
Manual ini dibuat khusus untuk Admin yang bertanggung jawab mengelola dokumen GCG untuk subdirektorat tertentu, termasuk upload dokumen, monitoring progress, dan manajemen dokumen pribadi.

\subsection{Hak Akses Admin}
Sebagai Admin, Anda memiliki akses ke:
\begin{itemize}
\item Dashboard Admin (overview personal)
\item Upload Dokumen (untuk subdirektorat yang ditugaskan)
\item Monitoring Progress (dokumen pribadi)
\item Manajemen Profil
\end{itemize}

\subsection{Perbedaan dengan Super Admin}
\begin{itemize}
\item \textbf{Admin}: Hanya dapat mengelola dokumen untuk subdirektorat yang ditugaskan
\item \textbf{Super Admin}: Dapat mengelola seluruh sistem dan semua subdirektorat
\item \textbf{Admin}: Tidak dapat mengubah konfigurasi sistem
\item \textbf{Super Admin}: Dapat mengubah semua konfigurasi sistem
\end{itemize}

% Login and Authentication
\section{Login dan Autentikasi}

\subsection{Langkah Login}
\begin{enumerate}
\item \textbf{Buka Browser} dan akses URL aplikasi
\item \textbf{Halaman Login} akan muncul
\item \textbf{Masukkan kredensial Admin}: Email/Username dan Password
\item \textbf{Klik tombol "Login"}
\item \textbf{Sistem akan memverifikasi} dan mengarahkan ke Dashboard Admin
\end{enumerate}

\subsection{Placeholder Gambar}
\begin{figure}[H]
\centering
\includegraphics[width=0.8\textwidth]{placeholder_login.png}
\caption{Halaman Login - Form login dengan field email dan password}
\label{fig:login}
\end{figure}

\subsection{Catatan Keamanan}
\begin{itemize}
\item Jangan bagikan kredensial Admin
\item Logout setelah selesai menggunakan aplikasi
\item Gunakan password yang kuat dan unik
\item Laporkan jika ada aktivitas mencurigakan
\item Hanya upload dokumen yang sesuai dengan subdirektorat yang ditugaskan
\end{itemize}

% Admin Menu
\section{Menu Admin}

\subsection{Dashboard Admin}

\subsubsection{Deskripsi}
Dashboard Admin menampilkan informasi dan statistik dokumen sesuai dengan subdirektorat yang ditugaskan. Dashboard ini memberikan overview personal tentang progress upload dokumen, checklist yang harus diselesaikan, dan statistik performa individual.

\subsubsection{Fitur Utama}
\begin{itemize}
\item \textbf{Statistik Pribadi}: Statistik dokumen yang diupload oleh admin
\item \textbf{Daftar Dokumen}: Daftar dokumen yang telah diupload
\item \textbf{Upload Dokumen}: Fitur untuk mengupload dokumen baru
\item \textbf{Progress Tracking}: Melacak progress upload dokumen
\item \textbf{Notifikasi}: Alert untuk dokumen yang harus diupload
\item \textbf{Quick Actions}: Akses cepat ke fitur utama
\end{itemize}

\subsubsection{Langkah Penggunaan Detail}
\begin{enumerate}
\item \textbf{Klik menu "Dashboard"} di sidebar (akan otomatis ke Dashboard Admin)
\item \textbf{Pilih tahun buku} dari dropdown tahun di bagian atas
\item \textbf{Lihat statistik pribadi}:
\begin{itemize}
\item Total dokumen yang harus diupload untuk subdirektorat Anda
\item Jumlah dokumen yang sudah diupload
\item Jumlah dokumen yang masih pending
\item Persentase completion rate personal
\item Progress chart individual
\end{itemize}
\item \textbf{Lihat daftar dokumen} yang telah diupload:
\begin{itemize}
\item Tabel dokumen dengan status terbaru
\item Filter berdasarkan status dan tanggal
\item Akses cepat untuk download atau edit
\end{itemize}
\item \textbf{Upload dokumen baru}:
\begin{itemize}
\item Klik tombol "Upload Dokumen" di bagian Quick Actions
\item Pilih checklist yang ingin diupload
\item Ikuti proses upload yang akan dijelaskan di bagian selanjutnya
\end{itemize}
\item \textbf{Download dokumen} jika diperlukan:
\begin{itemize}
\item Klik ikon download pada dokumen yang ingin diunduh
\item File akan terunduh dengan nama asli
\end{itemize}
\end{enumerate}

\subsubsection{Interpretasi Data Dashboard}
\begin{itemize}
\item \textbf{Completion Rate}: Persentase dokumen yang sudah diupload untuk subdirektorat Anda
\item \textbf{Green Status}: Dokumen sudah diupload dan diverifikasi
\item \textbf{Yellow Status}: Dokumen sedang dalam proses review
\item \textbf{Red Status}: Dokumen belum diupload atau terlambat
\item \textbf{Progress Bar}: Menunjukkan progress upload dokumen
\end{itemize}

\subsubsection{Placeholder Gambar}
\begin{figure}[H]
\centering
\includegraphics[width=0.9\textwidth]{placeholder_dashboard_admin.png}
\caption{Dashboard Admin - Statistik pribadi, tombol upload, dan daftar dokumen}
\label{fig:dashboard_admin}
\end{figure}

\subsection{Upload Dokumen}

\subsubsection{Deskripsi}
Fitur upload dokumen memungkinkan Admin untuk mengunggah dokumen GCG sesuai dengan checklist yang telah ditentukan untuk subdirektorat yang ditugaskan.

\subsubsection{Fitur Utama}
\begin{itemize}
\item \textbf{Pilih Checklist}: Memilih dokumen yang akan diupload
\item \textbf{Upload File}: Mengunggah file dokumen
\item \textbf{Validasi File}: Memverifikasi format dan ukuran file
\item \textbf{Status Tracking}: Melacak status upload
\item \textbf{Preview File}: Melihat preview file sebelum upload
\end{itemize}

\subsubsection{Langkah Penggunaan Detail}
\begin{enumerate}
\item \textbf{Klik tombol "Upload Dokumen"} di Dashboard Admin
\item \textbf{Dialog upload akan terbuka} dengan informasi:
\begin{itemize}
\item Daftar checklist yang tersedia untuk subdirektorat Anda
\item Status setiap checklist (Sudah diupload/Belum diupload)
\item Format file yang diperbolehkan
\item Ukuran file maksimal
\end{itemize}
\item \textbf{Pilih checklist} yang ingin diupload:
\begin{itemize}
\item Klik pada checklist yang ingin diupload
\item Pastikan checklist belum diupload sebelumnya
\item Baca deskripsi checklist dengan teliti
\end{itemize}
\item \textbf{Pilih file} dokumen yang akan diupload:
\begin{itemize}
\item Klik tombol "Browse" atau "Choose File"
\item Pilih file dari komputer (PDF, DOC, DOCX)
\item Pastikan ukuran file maksimal 10MB
\item Pastikan nama file tidak mengandung karakter khusus
\end{itemize}
\item \textbf{Verifikasi file}:
\begin{itemize}
\item Sistem akan menampilkan preview file
\item Periksa nama file dan ukuran
\item Pastikan file sesuai dengan checklist
\item Pastikan file tidak corrupt atau rusak
\end{itemize}
\item \textbf{Klik "Upload"} untuk memulai proses upload
\item \textbf{Tunggu proses upload} selesai:
\begin{itemize}
\item Progress bar akan menampilkan status upload
\item Jangan tutup browser selama proses upload
\item Sistem akan memvalidasi file setelah upload
\end{itemize}
\item \textbf{Konfirmasi upload} berhasil:
\begin{itemize}
\item Notifikasi sukses akan muncul
\item Status dokumen berubah menjadi "Review"
\item File akan tersedia untuk download
\item Progress chart akan terupdate
\end{itemize}
\end{enumerate}

\subsubsection{Tips Upload Dokumen}
\begin{itemize}
\item \textbf{Format File}: Gunakan format PDF untuk dokumen resmi
\item \textbf{Nama File}: Gunakan nama file yang deskriptif dan konsisten
\item \textbf{Ukuran File}: Pastikan file tidak melebihi 10MB
\item \textbf{Kualitas File}: Pastikan file tidak blur atau rusak
\item \textbf{Backup File}: Simpan salinan file di komputer lokal
\item \textbf{Upload Tepat Waktu}: Upload dokumen sesuai jadwal yang ditentukan
\end{itemize}

\subsubsection{Placeholder Gambar}
\begin{figure}[H]
\centering
\includegraphics[width=0.7\textwidth]{placeholder_dialog_upload_admin.png}
\caption{Dialog Upload Dokumen Admin - Form upload dengan pilihan checklist}
\label{fig:dialog_upload_admin}
\end{figure}

\begin{figure}[H]
\centering
\includegraphics[width=0.8\textwidth]{placeholder_progress_upload.png}
\caption{Progress Upload - Progress bar dan status upload dokumen}
\label{fig:progress_upload}
\end{figure}

\subsection{Monitoring Progress}

\subsubsection{Deskripsi}
Fitur monitoring progress memungkinkan Admin untuk melacak status dan progress upload dokumen yang telah dilakukan.

\subsubsection{Fitur Utama}
\begin{itemize}
\item \textbf{Status Dokumen}: Melihat status setiap dokumen yang diupload
\item \textbf{Progress Chart}: Visualisasi progress upload
\item \textbf{History Upload}: Riwayat upload dokumen
\item \textbf{Filter Status}: Menyaring dokumen berdasarkan status
\item \textbf{Export Progress}: Mengunduh laporan progress
\end{itemize}

\subsubsection{Langkah Penggunaan Detail}
\begin{enumerate}
\item \textbf{Lihat status dokumen} di Dashboard Admin:
\begin{itemize}
\item Status "Pending": Dokumen belum diupload
\item Status "Review": Dokumen sedang dalam proses validasi
\item Status "Approved": Dokumen telah disetujui
\item Status "Rejected": Dokumen ditolak dengan alasan
\end{itemize}
\item \textbf{Lihat progress chart}:
\begin{itemize}
\item Grafik batang menunjukkan progress per aspek GCG
\item Grafik pie menunjukkan distribusi status dokumen
\item Progress bar menunjukkan completion rate
\end{itemize}
\item \textbf{Filter dokumen} berdasarkan status:
\begin{itemize}
\item Klik filter dropdown
\item Pilih status yang ingin dilihat
\item Tabel akan menampilkan dokumen sesuai filter
\end{itemize}
\item \textbf{Lihat history upload}:
\begin{itemize}
\item Klik tab "History" di Dashboard
\item Lihat riwayat upload dokumen
\item Lihat tanggal upload dan status terbaru
\end{itemize}
\item \textbf{Export progress report}:
\begin{itemize}
\item Klik tombol "Export Progress"
\item Pilih format file (PDF, Excel)
\item Laporan akan terunduh otomatis
\end{itemize}
\end{enumerate}

\subsubsection{Placeholder Gambar}
\begin{figure}[H]
\centering
\includegraphics[width=0.9\textwidth]{placeholder_monitoring_progress.png}
\caption{Monitoring Progress - Chart progress dan status dokumen}
\label{fig:monitoring_progress}
\end{figure}

% General Features
\section{Fitur Umum}

\subsection{Manajemen Profil}

\subsubsection{Deskripsi}
Fitur manajemen profil memungkinkan Admin untuk mengelola informasi pribadi, mengubah password, dan mengatur preferensi akun.

\subsubsection{Langkah Penggunaan}
\begin{enumerate}
\item \textbf{Klik nama pengguna} di pojok kanan atas halaman
\item \textbf{Pilih "Profil"} dari dropdown menu yang muncul
\item \textbf{Edit informasi} yang diperlukan:
\begin{itemize}
\item Nama lengkap
\item Email
\item Nomor telepon
\item Subdirektorat yang ditugaskan
\item Divisi (jika ada)
\item Foto profil (opsional)
\end{itemize}
\item \textbf{Ubah password} jika diperlukan:
\begin{itemize}
\item Masukkan password lama
\item Masukkan password baru
\item Konfirmasi password baru
\item Klik "Ubah Password"
\end{itemize}
\item \textbf{Klik "Simpan"} untuk menyimpan perubahan
\item \textbf{Sistem akan menampilkan notifikasi} sukses
\end{enumerate}

\subsubsection{Informasi yang Dapat Diubah}
\begin{itemize}
\item \textbf{Informasi Pribadi}: Nama, email, nomor telepon
\item \textbf{Informasi Profesional}: Subdirektorat, divisi
\item \textbf{Preferensi}: Bahasa, timezone, notifikasi
\item \textbf{Keamanan}: Password, pertanyaan keamanan
\end{itemize}

\subsubsection{Placeholder Gambar}
\begin{figure}[H]
\centering
\includegraphics[width=0.6\textwidth]{placeholder_menu_profil.png}
\caption{Menu Profil - Dropdown menu dengan opsi profil, pengaturan, dan logout}
\label{fig:menu_profil}
\end{figure}

\begin{figure}[H]
\centering
\includegraphics[width=0.7\textwidth]{placeholder_form_profil.png}
\caption{Form Profil - Form edit informasi profil dan password}
\label{fig:form_profil}
\end{figure}

\subsection{Logout}

\subsubsection{Deskripsi}
Fitur logout memungkinkan Admin untuk keluar dari sistem dengan aman dan mengakhiri sesi login.

\subsubsection{Langkah Penggunaan}
\begin{enumerate}
\item \textbf{Klik nama pengguna} di pojok kanan atas halaman
\item \textbf{Pilih "Logout"} dari dropdown menu
\item \textbf{Konfirmasi logout} di dialog yang muncul
\item \textbf{Sistem akan mengarahkan} ke halaman login
\item \textbf{Sesi akan berakhir} dan data sementara akan dihapus
\end{enumerate}

\subsection{Responsive Design}

\subsubsection{Fitur Mobile}
\begin{itemize}
\item \textbf{Sidebar Collapsible}: Sidebar dapat di-collapse di layar kecil
\item \textbf{Touch Friendly}: Tombol dan menu yang mudah disentuh
\item \textbf{Adaptive Layout}: Layout yang menyesuaikan ukuran layar
\item \textbf{Mobile Navigation}: Menu hamburger untuk navigasi mobile
\item \textbf{Responsive Tables}: Tabel yang dapat di-scroll horizontal
\end{itemize}

\subsubsection{Fitur Desktop}
\begin{itemize}
\item \textbf{Full Sidebar}: Sidebar selalu terbuka di layar besar
\item \textbf{Hover Effects}: Efek hover pada tombol dan menu
\item \textbf{Keyboard Navigation}: Navigasi menggunakan keyboard
\item \textbf{Multi-window Support}: Dapat membuka multiple tab
\item \textbf{Drag and Drop}: Upload file dengan drag and drop
\end{itemize}

% Troubleshooting
\section{Troubleshooting}

\subsection{Masalah Umum dan Solusi}

\subsubsection{Tidak Bisa Login}
\textbf{Gejala}: Error saat memasukkan username/password\\
\textbf{Kemungkinan Penyebab}:
\begin{itemize}
\item Username atau password salah
\item Akun terkunci karena terlalu banyak percobaan login
\item Koneksi internet bermasalah
\item Browser cache corrupt
\item Server sedang maintenance
\end{itemize}
\textbf{Solusi}:
\begin{enumerate}
\item Pastikan username dan password benar
\item Periksa koneksi internet
\item Clear cache browser (Ctrl+F5)
\item Coba browser lain
\item Hubungi Super Admin jika masalah berlanjut
\end{enumerate}

\subsubsection{File Upload Gagal}
\textbf{Gejala}: File tidak bisa diupload\\
\textbf{Kemungkinan Penyebab}:
\begin{itemize}
\item Ukuran file terlalu besar
\item Format file tidak didukung
\item Koneksi internet terputus
\item Server storage penuh
\item File corrupt atau rusak
\end{itemize}
\textbf{Solusi}:
\begin{enumerate}
\item Periksa ukuran file (maksimal 10MB)
\item Pastikan format file didukung (PDF, DOC, DOCX)
\item Periksa koneksi internet
\item Coba upload file lain
\item Kompres file jika terlalu besar
\item Hubungi Super Admin jika masalah berlanjut
\end{enumerate}

\subsubsection{Dokumen Tidak Muncul di Dashboard}
\textbf{Gejala}: Dokumen yang diupload tidak muncul di dashboard\\
\textbf{Kemungkinan Penyebab}:
\begin{itemize}
\item Upload belum selesai
\item Browser cache
\item Session expired
\item Server sync delay
\end{itemize}
\textbf{Solusi}:
\begin{enumerate}
\item Tunggu beberapa saat untuk sync
\item Refresh halaman (F5)
\item Logout dan login kembali
\item Clear browser cache
\item Periksa tahun buku yang dipilih
\end{enumerate}

\subsubsection{Status Dokumen Tidak Berubah}
\textbf{Gejala}: Status dokumen tetap "Review" padahal sudah lama\\
\textbf{Kemungkinan Penyebab}:
\begin{itemize}
\item Dokumen sedang dalam proses validasi
\item Super Admin belum melakukan review
\item Ada masalah dengan dokumen
\end{itemize}
\textbf{Solusi}:
\begin{enumerate}
\item Tunggu proses validasi selesai
\item Hubungi Super Admin untuk konfirmasi
\item Periksa kualitas dokumen yang diupload
\item Upload ulang jika diperlukan
\end{enumerate}

\subsection{Kontak Support}
Jika mengalami masalah yang tidak dapat diselesaikan:
\begin{itemize}
\item \textbf{Email}: support@gcg-system.com
\item \textbf{Telepon}: (021) 1234-5678
\item \textbf{WhatsApp}: +62 812-3456-7890
\item \textbf{Jam Operasional}: Senin - Jumat, 08:00 - 17:00 WIB
\item \textbf{Super Admin}: Hubungi Super Admin untuk masalah teknis
\end{itemize}

% Conclusion
\section{Kesimpulan}

Manual book ini telah menjelaskan secara lengkap cara menggunakan aplikasi GCG Documents Management System sebagai Admin. Dengan mengikuti panduan ini, Admin diharapkan dapat:

\begin{enumerate}
\item \textbf{Mengelola dokumen} GCG untuk subdirektorat yang ditugaskan
\item \textbf{Mengupload dokumen} sesuai dengan checklist yang ditentukan
\item \textbf{Memantau progress} upload dokumen pribadi
\item \textbf{Mengelola profil} dan preferensi akun
\item \textbf{Mengatasi masalah} umum yang mungkin terjadi
\end{enumerate}

\subsection{Tips Penggunaan Admin}
\begin{itemize}
\item \textbf{Upload tepat waktu}: Upload dokumen sesuai jadwal yang ditentukan
\item \textbf{Format file yang benar}: Gunakan format PDF untuk dokumen resmi
\item \textbf{Nama file yang jelas}: Gunakan nama file yang deskriptif
\item \textbf{Backup dokumen}: Simpan salinan dokumen di komputer lokal
\item \textbf{Periksa status}: Rutin cek status dokumen yang diupload
\item \textbf{Komunikasi}: Laporkan masalah ke Super Admin jika diperlukan
\end{itemize}

\subsection{Best Practices Admin}
\begin{itemize}
\item \textbf{Dokumen Management}: Kelola dokumen dengan sistem yang terorganisir
\item \textbf{Quality Control}: Pastikan kualitas dokumen yang diupload
\item \textbf{Timely Upload}: Upload dokumen sesuai jadwal yang ditentukan
\item \textbf{Communication}: Berkomunikasi dengan Super Admin jika ada masalah
\item \textbf{Data Backup}: Selalu backup dokumen penting
\item \textbf{Security}: Jaga keamanan akses dan data
\end{itemize}

\subsection{Peran dan Tanggung Jawab Admin}
\begin{itemize}
\item \textbf{Upload Dokumen}: Mengupload dokumen GCG sesuai checklist
\item \textbf{Quality Assurance}: Memastikan kualitas dokumen yang diupload
\item \textbf{Progress Reporting}: Melaporkan progress upload ke Super Admin
\item \textbf{Data Management}: Mengelola dokumen dengan baik
\item \textbf{Compliance}: Memastikan dokumen sesuai dengan standar GCG
\end{itemize}

% Appendix
\appendix
\section{Glossary}
\begin{description}
\item[GCG] Good Corporate Governance - Tata kelola perusahaan yang baik
\item[Admin] Pengguna dengan hak akses terbatas untuk subdirektorat tertentu
\item[Checklist] Daftar dokumen yang harus diupload untuk setiap aspek GCG
\item[Upload] Proses mengunggah dokumen ke sistem
\item[Download] Proses mengunduh dokumen dari sistem
\item[Validasi] Proses verifikasi kelengkapan dan format dokumen
\item[Review] Proses peninjauan dokumen oleh Super Admin
\item[Approved] Status dokumen yang telah disetujui
\item[Pending] Status dokumen yang belum diupload
\item[Progress] Kemajuan upload dokumen
\end{description}

\section{FAQ (Frequently Asked Questions)}

\subsection{Pertanyaan Umum Admin}
\begin{enumerate}
\item \textbf{Q: Bagaimana cara mengupload dokumen?}\\
A: Klik tombol "Upload Dokumen" di Dashboard, pilih checklist, lalu pilih file yang akan diupload.

\item \textbf{Q: Format file apa saja yang didukung?}\\
A: Sistem mendukung format PDF, DOC, dan DOCX dengan ukuran maksimal 10MB.

\item \textbf{Q: Bagaimana cara melihat status dokumen?}\\
A: Status dokumen dapat dilihat di Dashboard Admin atau tab "History".

\item \textbf{Q: Apa yang harus dilakukan jika upload gagal?}\\
A: Periksa ukuran file, format file, dan koneksi internet. Jika masih gagal, hubungi Super Admin.

\item \textbf{Q: Bagaimana cara mengubah password?}\\
A: Klik nama pengguna di pojok kanan atas, pilih "Profil", lalu klik "Ubah Password".

\item \textbf{Q: Bagaimana cara mengunduh dokumen yang sudah diupload?}\\
A: Klik ikon download pada dokumen yang ingin diunduh di Dashboard Admin.

\item \textbf{Q: Apa yang harus dilakukan jika status dokumen tetap "Review"?}\\
A: Tunggu proses validasi selesai atau hubungi Super Admin untuk konfirmasi.

\item \textbf{Q: Bagaimana cara melihat progress upload?}\\
A: Progress upload dapat dilihat di Dashboard Admin melalui progress chart dan statistik.
\end{enumerate}

\section{Changelog}
\begin{itemize}
\item \textbf{Versi 1.0 (Desember 2024)}: Manual book Admin pertama dengan fitur lengkap
\end{itemize}

\end{document}
