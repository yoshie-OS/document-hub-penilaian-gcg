\documentclass[12pt,a4paper]{article}
\usepackage[utf8]{inputenc}
\usepackage[T1]{fontenc}
\usepackage[english,indonesian]{babel}
\usepackage{geometry}
\usepackage{graphicx}
\usepackage{hyperref}
\usepackage{enumitem}
\usepackage{fancyhdr}
\usepackage{titlesec}
\usepackage{color}
\usepackage{float}

% Page setup
\geometry{
    left=2.5cm,
    right=2.5cm,
    top=2.5cm,
    bottom=2.5cm
}

% Hyperlink setup
\hypersetup{
    colorlinks=true,
    linkcolor=blue,
    filecolor=magenta,
    urlcolor=blue,
    citecolor=blue
}

% Header and footer setup
\pagestyle{fancy}
\fancyhf{}
\fancyhead[L]{\textbf{Manual Book Aplikasi GCG}}
\fancyhead[R]{\thepage}
\renewcommand{\headrulewidth}{0.4pt}

% Title formatting
\titleformat{\section}{\Large\bfseries\color{blue}}{\thesection}{1em}{}
\titleformat{\subsection}{\large\bfseries\color{darkgray}}{\thesubsection}{1em}{}
\titleformat{\subsubsection}{\normalsize\bfseries}{\thesubsubsection}{1em}{}

% Custom colors
\definecolor{lightblue}{RGB}{173,216,230}
\definecolor{darkblue}{RGB}{0,0,139}
\definecolor{darkgray}{RGB}{64,64,64}

\begin{document}

% Title page
\begin{titlepage}
    \centering
    \vspace*{2cm}
    
    {\Huge\bfseries\color{darkblue} MANUAL BOOK APLIKASI\\[0.5cm] GOOD CORPORATE GOVERNANCE (GCG)\\[0.5cm] DOCUMENTS MANAGEMENT SYSTEM}
    
    \vspace{2cm}
    
    \includegraphics[width=0.3\textwidth]{placeholder_logo.png}
    
    \vspace{2cm}
    
    {\Large\bfseries Panduan Penggunaan Lengkap}
    
    \vspace{1cm}
    
    {\large Untuk Super Admin, Admin, dan User}
    
    \vfill
    
    {\large \textbf{Dibuat oleh:} Tim Pengembangan Aplikasi GCG}
    
    \vspace{0.5cm}
    
    {\large \textbf{Versi:} 1.0}
    
    \vspace{0.5cm}
    
    {\large \textbf{Tanggal:} Desember 2024}
    
    \vspace{0.5cm}
    
    {\large \textbf{Status:} Final}
    
    \vspace{1cm}
    
\end{titlepage}

% Table of contents
\tableofcontents
\newpage

% Introduction
\section{Pendahuluan}

\subsection{Deskripsi Aplikasi}
Aplikasi Good Corporate Governance (GCG) Documents Management System adalah sistem manajemen dokumen yang dirancang untuk mengelola dokumen-dokumen terkait tata kelola perusahaan yang baik. Aplikasi ini memiliki dua level pengguna utama:

\begin{itemize}
    \item \textbf{Super Admin}: Memiliki akses penuh ke semua fitur dan pengaturan sistem
    \item \textbf{Admin}: Memiliki akses terbatas untuk mengelola dokumen sesuai dengan direktorat/subdirektorat yang ditugaskan
\end{itemize}

\subsection{Tujuan Manual}
Manual ini dibuat untuk membantu pengguna memahami cara menggunakan aplikasi GCG secara efektif dan efisien.

\newpage

% Application Structure
\section{Struktur Aplikasi}

\subsection{Hierarki Pengguna}
\begin{verbatim}
Super Admin (Level Tertinggi)
├── Pengaturan Baru
├── Dashboard
├── Monitoring & Upload GCG
├── Arsip Dokumen
├── Performa GCG
└── AOI Management

Admin (Level Menengah)
└── Dashboard

User (Level Dasar)
└── Dashboard
\end{verbatim}

\subsection{Fitur Utama}
\begin{itemize}
    \item Manajemen dokumen GCG
    \item Monitoring upload dokumen
    \item Pengaturan aspek dan checklist
    \item Manajemen AOI (Area of Improvement)
    \item Arsip dokumen
    \item Dashboard statistik
\end{itemize}

\newpage

% Login and Authentication
\section{Login dan Autentikasi}

\subsection{Langkah Login}
\begin{enumerate}
    \item \textbf{Buka Browser} dan akses URL aplikasi
    \item \textbf{Halaman Login} akan muncul
    \item \textbf{Masukkan kredensial}:
    \begin{itemize}
        \item Email/Username
        \item Password
    \end{itemize}
    \item \textbf{Klik tombol "Login"}
    \item \textbf{Sistem akan memverifikasi} dan mengarahkan ke dashboard sesuai role
\end{enumerate}

\subsection{Placeholder Gambar}
\begin{figure}[H]
    \centering
    \includegraphics[width=0.8\textwidth]{placeholder_login.png}
    \caption{Halaman Login - Form login dengan field email dan password}
    \label{fig:login}
\end{figure}

\subsection{Catatan Keamanan}
\begin{itemize}
    \item Jangan bagikan kredensial login
    \item Logout setelah selesai menggunakan aplikasi
    \item Gunakan password yang kuat
\end{itemize}

\newpage

% Super Admin Menu
\section{Menu Super Admin}

\subsection{Pengaturan Baru}

\subsubsection{Deskripsi}
Menu ini berfungsi untuk mengatur konfigurasi dasar sistem, termasuk manajemen tahun buku, struktur perusahaan, dan checklist GCG.

\subsubsection{Fitur Utama}
\begin{itemize}
    \item \textbf{Manajemen Tahun Buku}: Menambah, mengedit, dan menghapus tahun buku
    \item \textbf{Struktur Perusahaan}: Mengatur hierarki direktorat, subdirektorat, dan divisi
    \item \textbf{Checklist GCG}: Membuat dan mengatur checklist dokumen yang diperlukan
    \item \textbf{Manajemen Aspek}: Mengatur aspek-aspek GCG yang akan dimonitor
\end{itemize}

\subsubsection{Langkah Penggunaan}

\paragraph{A. Manajemen Tahun Buku}
\begin{enumerate}
    \item \textbf{Klik menu "Pengaturan Baru"}
    \item \textbf{Pilih tab "Tahun Buku"}
    \item \textbf{Untuk menambah tahun baru}:
    \begin{itemize}
        \item Klik tombol "Tambah Tahun Baru"
        \item Masukkan tahun (contoh: 2024)
        \item Klik "Simpan"
    \end{itemize}
    \item \textbf{Untuk mengedit tahun}:
    \begin{itemize}
        \item Klik ikon edit pada tahun yang ingin diubah
        \item Ubah tahun sesuai kebutuhan
        \item Klik "Simpan"
    \end{itemize}
    \item \textbf{Untuk menghapus tahun}:
    \begin{itemize}
        \item Klik ikon hapus pada tahun yang ingin dihapus
        \item Konfirmasi penghapusan
    \end{itemize}
\end{enumerate}

\paragraph{B. Struktur Perusahaan}
\begin{enumerate}
    \item \textbf{Pilih tab "Struktur Perusahaan"}
    \item \textbf{Manajemen Direktorat}:
    \begin{itemize}
        \item Klik "Tambah Direktorat" untuk menambah direktorat baru
        \item Isi nama direktorat dan tahun aktif
        \item Klik "Simpan"
    \end{itemize}
    \item \textbf{Manajemen Subdirektorat}:
    \begin{itemize}
        \item Pilih direktorat dari dropdown
        \item Klik "Tambah Subdirektorat"
        \item Isi nama subdirektorat
        \item Klik "Simpan"
    \end{itemize}
    \item \textbf{Manajemen Divisi}:
    \begin{itemize}
        \item Pilih subdirektorat dari dropdown
        \item Klik "Tambah Divisi"
        \item Isi nama divisi
        \item Klik "Simpan"
    \end{itemize}
\end{enumerate}

\paragraph{C. Checklist GCG}
\begin{enumerate}
    \item \textbf{Pilih tab "Checklist GCG"}
    \item \textbf{Untuk menambah checklist baru}:
    \begin{itemize}
        \item Klik "Tambah Checklist"
        \item Pilih tahun buku
        \item Pilih aspek GCG
        \item Isi deskripsi checklist
        \item Klik "Simpan"
    \end{itemize}
    \item \textbf{Untuk mengedit checklist}:
    \begin{itemize}
        \item Klik ikon edit pada checklist yang ingin diubah
        \item Ubah informasi sesuai kebutuhan
        \item Klik "Simpan"
    \end{itemize}
    \item \textbf{Untuk menghapus checklist}:
    \begin{itemize}
        \item Klik ikon hapus pada checklist yang ingin dihapus
        \item Konfirmasi penghapusan
    \end{itemize}
\end{enumerate}

\subsubsection{Placeholder Gambar}
\begin{figure}[H]
    \centering
    \includegraphics[width=0.9\textwidth]{placeholder_pengaturan_baru.png}
    \caption{Halaman Pengaturan Baru - 4 tab utama: Tahun Buku, Struktur Perusahaan, Checklist GCG, dan Manajemen Aspek}
    \label{fig:pengaturan_baru}
\end{figure}

\begin{figure}[H]
    \centering
    \includegraphics[width=0.8\textwidth]{placeholder_tahun_buku.png}
    \caption{Tab Tahun Buku - Tabel tahun buku dengan tombol tambah, edit, dan hapus}
    \label{fig:tahun_buku}
\end{figure}

\begin{figure}[H]
    \centering
    \includegraphics[width=0.8\textwidth]{placeholder_struktur_perusahaan.png}
    \caption{Tab Struktur Perusahaan - Hierarki direktorat, subdirektorat, dan divisi}
    \label{fig:struktur_perusahaan}
\end{figure}

\begin{figure}[H]
    \centering
    \includegraphics[width=0.8\textwidth]{placeholder_checklist_gcg.png}
    \caption{Tab Checklist GCG - Tabel checklist dengan tombol tambah, edit, dan hapus}
    \label{fig:checklist_gcg}
\end{figure}

\end{document}
