\documentclass[12pt,a4paper]{article}
\usepackage[utf8]{inputenc}
\usepackage[T1]{fontenc}
\usepackage[english,indonesian]{babel}
\usepackage{geometry}
\usepackage{graphicx}
\usepackage{hyperref}
\usepackage{enumitem}
\usepackage{fancyhdr}
\usepackage{titlesec}
\usepackage{color}
\usepackage{float}

% Page setup
\geometry{
    left=2.5cm,
    right=2.5cm,
    top=2.5cm,
    bottom=2.5cm
}

% Hyperlink setup
\hypersetup{
    colorlinks=true,
    linkcolor=blue,
    filecolor=magenta,
    urlcolor=blue,
    citecolor=blue
}

% Header and footer setup
\pagestyle{fancy}
\fancyhf{}
\fancyhead[L]{\textbf{Manual Book Aplikasi GCG}}
\fancyhead[R]{\thepage}
\renewcommand{\headrulewidth}{0.4pt}

% Title formatting
\titleformat{\section}{\Large\bfseries\color{blue}}{\thesection}{1em}{}
\titleformat{\subsection}{\large\bfseries\color{darkgray}}{\thesubsection}{1em}{}
\titleformat{\subsubsection}{\normalsize\bfseries}{\thesubsubsection}{1em}{}

% Custom colors
\definecolor{lightblue}{RGB}{173,216,230}
\definecolor{darkblue}{RGB}{0,0,139}
\definecolor{darkgray}{RGB}{64,64,64}

\begin{document}

% Title page
\begin{titlepage}
    \centering
    \vspace*{2cm}
    
    {\Huge\bfseries\color{darkblue} MANUAL BOOK APLIKASI\\[0.5cm] GOOD CORPORATE GOVERNANCE (GCG)\\[0.5cm] DOCUMENTS MANAGEMENT SYSTEM}
    
    \vspace{2cm}
    
    \includegraphics[width=0.3\textwidth]{placeholder_logo.png}
    
    \vspace{2cm}
    
    {\Large\bfseries Panduan Penggunaan Lengkap}
    
    \vspace{1cm}
    
    {\large Untuk Super Admin, Admin, dan User}
    
    \vfill
    
    {\large \textbf{Dibuat oleh:} Tim Pengembangan Aplikasi GCG}
    
    \vspace{0.5cm}
    
    {\large \textbf{Versi:} 1.0}
    
    \vspace{0.5cm}
    
    {\large \textbf{Tanggal:} Desember 2024}
    
    \vspace{0.5cm}
    
    {\large \textbf{Status:} Final}
    
    \vspace{1cm}
    
\end{titlepage}

% Table of contents
\tableofcontents
\newpage

% Introduction
\section{Pendahuluan}

\subsection{Deskripsi Aplikasi}
Aplikasi Good Corporate Governance (GCG) Documents Management System adalah sistem manajemen dokumen yang dirancang untuk mengelola dokumen-dokumen terkait tata kelola perusahaan yang baik. Aplikasi ini memiliki dua level pengguna utama:

\begin{itemize}
    \item \textbf{Super Admin}: Memiliki akses penuh ke semua fitur dan pengaturan sistem
    \item \textbf{Admin}: Memiliki akses terbatas untuk mengelola dokumen sesuai dengan direktorat/subdirektorat yang ditugaskan
\end{itemize}

\subsection{Tujuan Manual}
Manual ini dibuat untuk membantu pengguna memahami cara menggunakan aplikasi GCG secara efektif dan efisien.

\newpage

% Application Structure
\section{Struktur Aplikasi}

\subsection{Hierarki Pengguna}
\begin{verbatim}
Super Admin (Level Tertinggi)
├── Pengaturan Baru
├── Dashboard
├── Monitoring & Upload GCG
├── Arsip Dokumen
├── Performa GCG
└── AOI Management

Admin (Level Menengah)
└── Dashboard

User (Level Dasar)
└── Dashboard
\end{verbatim}

\subsection{Fitur Utama}
\begin{itemize}
    \item Manajemen dokumen GCG
    \item Monitoring upload dokumen
    \item Pengaturan aspek dan checklist
    \item Manajemen AOI (Area of Improvement)
    \item Arsip dokumen
    \item Dashboard statistik
\end{itemize}

\newpage

% Login and Authentication
\section{Login dan Autentikasi}

\subsection{Langkah Login}
\begin{enumerate}
    \item \textbf{Buka Browser} dan akses URL aplikasi
    \item \textbf{Halaman Login} akan muncul
    \item \textbf{Masukkan kredensial}:
    \begin{itemize}
        \item Email/Username
        \item Password
    \end{itemize}
    \item \textbf{Klik tombol "Login"}
    \item \textbf{Sistem akan memverifikasi} dan mengarahkan ke dashboard sesuai role
\end{enumerate}

\subsection{Placeholder Gambar}
\begin{figure}[H]
    \centering
    \includegraphics[width=0.8\textwidth]{placeholder_login.png}
    \caption{Halaman Login - Form login dengan field email dan password}
    \label{fig:login}
\end{figure}

\subsection{Catatan Keamanan}
\begin{itemize}
    \item Jangan bagikan kredensial login
    \item Logout setelah selesai menggunakan aplikasi
    \item Gunakan password yang kuat
\end{itemize}

\newpage

% Super Admin Menu
\section{Menu Super Admin}

\subsection{Pengaturan Baru}

\subsubsection{Deskripsi}
Menu ini berfungsi untuk mengatur konfigurasi dasar sistem, termasuk manajemen tahun buku, struktur perusahaan, dan checklist GCG.

\subsubsection{Fitur Utama}
\begin{itemize}
    \item \textbf{Manajemen Tahun Buku}: Menambah, mengedit, dan menghapus tahun buku
    \item \textbf{Struktur Perusahaan}: Mengatur hierarki direktorat, subdirektorat, dan divisi
    \item \textbf{Checklist GCG}: Membuat dan mengatur checklist dokumen yang diperlukan
    \item \textbf{Manajemen Aspek}: Mengatur aspek-aspek GCG yang akan dimonitor
\end{itemize}

\subsubsection{Langkah Penggunaan}

\paragraph{A. Manajemen Tahun Buku}
\begin{enumerate}
    \item \textbf{Klik menu "Pengaturan Baru"}
    \item \textbf{Pilih tab "Tahun Buku"}
    \item \textbf{Untuk menambah tahun baru}:
    \begin{itemize}
        \item Klik tombol "Tambah Tahun Baru"
        \item Masukkan tahun (contoh: 2024)
        \item Klik "Simpan"
    \end{itemize}
    \item \textbf{Untuk mengedit tahun}:
    \begin{itemize}
        \item Klik ikon edit pada tahun yang ingin diubah
        \item Ubah tahun sesuai kebutuhan
        \item Klik "Simpan"
    \end{itemize}
    \item \textbf{Untuk menghapus tahun}:
    \begin{itemize}
        \item Klik ikon hapus pada tahun yang ingin dihapus
        \item Konfirmasi penghapusan
    \end{itemize}
\end{enumerate}

\paragraph{B. Struktur Perusahaan}
\begin{enumerate}
    \item \textbf{Pilih tab "Struktur Perusahaan"}
    \item \textbf{Manajemen Direktorat}:
    \begin{itemize}
        \item Klik "Tambah Direktorat" untuk menambah direktorat baru
        \item Isi nama direktorat dan tahun aktif
        \item Klik "Simpan"
    \end{itemize}
    \item \textbf{Manajemen Subdirektorat}:
    \begin{itemize}
        \item Pilih direktorat dari dropdown
        \item Klik "Tambah Subdirektorat"
        \item Isi nama subdirektorat
        \item Klik "Simpan"
    \end{itemize}
    \item \textbf{Manajemen Divisi}:
    \begin{itemize}
        \item Pilih subdirektorat dari dropdown
        \item Klik "Tambah Divisi"
        \item Isi nama divisi
        \item Klik "Simpan"
    \end{itemize}
\end{enumerate}

\paragraph{C. Checklist GCG}
\begin{enumerate}
    \item \textbf{Pilih tab "Checklist GCG"}
    \item \textbf{Untuk menambah checklist baru}:
    \begin{itemize}
        \item Klik "Tambah Checklist"
        \item Pilih tahun buku
        \item Pilih aspek GCG
        \item Isi deskripsi checklist
        \item Klik "Simpan"
    \end{itemize}
    \item \textbf{Untuk mengedit checklist}:
    \begin{itemize}
        \item Klik ikon edit pada checklist yang ingin diubah
        \item Ubah informasi sesuai kebutuhan
        \item Klik "Simpan"
    \end{itemize}
    \item \textbf{Untuk menghapus checklist}:
    \begin{itemize}
        \item Klik ikon hapus pada checklist yang ingin dihapus
        \item Konfirmasi penghapusan
    \end{itemize}
\end{enumerate}

\subsubsection{Placeholder Gambar}
\begin{figure}[H]
    \centering
    \includegraphics[width=0.9\textwidth]{placeholder_pengaturan_baru.png}
    \caption{Halaman Pengaturan Baru - 4 tab utama: Tahun Buku, Struktur Perusahaan, Checklist GCG, dan Manajemen Aspek}
    \label{fig:pengaturan_baru}
\end{figure}

\begin{figure}[H]
    \centering
    \includegraphics[width=0.8\textwidth]{placeholder_tahun_buku.png}
    \caption{Tab Tahun Buku - Tabel tahun buku dengan tombol tambah, edit, dan hapus}
    \label{fig:tahun_buku}
\end{figure}

\begin{figure}[H]
    \centering
    \includegraphics[width=0.8\textwidth]{placeholder_struktur_perusahaan.png}
    \caption{Tab Struktur Perusahaan - Hierarki direktorat, subdirektorat, dan divisi}
    \label{fig:struktur_perusahaan}
\end{figure}

\begin{figure}[H]
    \centering
    \includegraphics[width=0.8\textwidth]{placeholder_checklist_gcg.png}
    \caption{Tab Checklist GCG - Tabel checklist dengan tombol tambah, edit, dan hapus}
    \label{fig:checklist_gcg}
\end{figure}

\newpage

\subsection{Dashboard}

\subsubsection{Deskripsi}
Dashboard Super Admin menampilkan overview statistik dan performa sistem secara keseluruhan.

\subsubsection{Fitur Utama}
\begin{itemize}
    \item \textbf{Statistik Dokumen}: Total dokumen, dokumen selesai, dan dokumen pending
    \item \textbf{Grafik Performa}: Visualisasi data dalam bentuk chart dan grafik
    \item \textbf{Trend Bulanan}: Perkembangan upload dokumen per bulan
    \item \textbf{Spider Chart}: Radar chart untuk aspek GCG
\end{itemize}

\subsubsection{Langkah Penggunaan}
\begin{enumerate}
    \item \textbf{Klik menu "Dashboard"}
    \item \textbf{Pilih tahun buku} dari dropdown tahun
    \item \textbf{Lihat statistik} yang ditampilkan
    \item \textbf{Analisis grafik} untuk memahami trend dan performa
\end{enumerate}

\subsubsection{Placeholder Gambar}
\begin{figure}[H]
    \centering
    \includegraphics[width=0.9\textwidth]{placeholder_dashboard_superadmin.png}
    \caption{Dashboard Super Admin - Statistik dokumen, grafik performa, dan chart}
    \label{fig:dashboard_superadmin}
\end{figure}

\newpage

\subsection{Monitoring \& Upload GCG}

\subsubsection{Deskripsi}
Menu ini berfungsi untuk memonitor status upload dokumen GCG dari berbagai subdirektorat dan melakukan upload dokumen.

\subsubsection{Fitur Utama}
\begin{itemize}
    \item \textbf{Monitoring Status}: Melihat status upload dokumen per subdirektorat
    \item \textbf{Upload Dokumen}: Mengunggah dokumen GCG
    \item \textbf{Filter dan Pencarian}: Mencari dokumen berdasarkan kriteria tertentu
    \item \textbf{Download Dokumen}: Mengunduh dokumen yang telah diupload
\end{itemize}

\subsubsection{Langkah Penggunaan}

\paragraph{A. Monitoring Status}
\begin{enumerate}
    \item \textbf{Klik menu "Monitoring \& Upload GCG"}
    \item \textbf{Pilih tahun buku} dari dropdown
    \item \textbf{Lihat tabel status} dokumen per subdirektorat
    \item \textbf{Gunakan filter} untuk melihat dokumen berdasarkan:
    \begin{itemize}
        \item Aspek GCG
        \item Status (Selesai/Pending)
        \item Subdirektorat
    \end{itemize}
    \item \textbf{Gunakan pencarian} untuk mencari dokumen tertentu
\end{enumerate}

\paragraph{B. Upload Dokumen}
\begin{enumerate}
    \item \textbf{Klik tombol "Upload Dokumen"} pada checklist yang ingin diupload
    \item \textbf{Pilih file} dokumen yang akan diupload
    \item \textbf{Pastikan format file} sesuai (PDF, DOC, DOCX)
    \item \textbf{Klik "Upload"}
    \item \textbf{Tunggu proses upload} selesai
    \item \textbf{Konfirmasi upload} berhasil
\end{enumerate}

\paragraph{C. Download Dokumen}
\begin{enumerate}
    \item \textbf{Klik ikon download} pada dokumen yang ingin diunduh
    \item \textbf{Pilih lokasi penyimpanan} di komputer
    \item \textbf{Konfirmasi download}
\end{enumerate}

\subsubsection{Placeholder Gambar}
\begin{figure}[H]
    \centering
    \includegraphics[width=0.9\textwidth]{placeholder_monitoring_upload.png}
    \caption{Halaman Monitoring \& Upload GCG - Tabel status dokumen dengan tombol upload dan download}
    \label{fig:monitoring_upload}
\end{figure}

\begin{figure}[H]
    \centering
    \includegraphics[width=0.7\textwidth]{placeholder_dialog_upload.png}
    \caption{Dialog Upload Dokumen - Form upload file dengan tombol browse dan upload}
    \label{fig:dialog_upload}
\end{figure}

\newpage

\subsection{Arsip Dokumen}

\subsubsection{Deskripsi}
Menu ini berfungsi untuk mengarsipkan dan mengelola dokumen GCG yang telah diupload.

\subsubsection{Fitur Utama}
\begin{itemize}
    \item \textbf{Arsip Dokumen}: Menyimpan dokumen dalam arsip terorganisir
    \item \textbf{Pencarian Arsip}: Mencari dokumen dalam arsip
    \item \textbf{Filter Arsip}: Menyaring dokumen berdasarkan kriteria
    \item \textbf{Download Arsip}: Mengunduh dokumen dari arsip
\end{itemize}

\subsubsection{Langkah Penggunaan}
\begin{enumerate}
    \item \textbf{Klik menu "Arsip Dokumen"}
    \item \textbf{Pilih tahun buku} dari dropdown
    \item \textbf{Gunakan filter} untuk menyaring dokumen:
    \begin{itemize}
        \item Direktorat
        \item Subdirektorat
        \item Aspek GCG
    \end{itemize}
    \item \textbf{Gunakan pencarian} untuk mencari dokumen tertentu
    \item \textbf{Klik ikon download} untuk mengunduh dokumen
    \item \textbf{Lihat detail dokumen} dengan mengklik nama file
\end{enumerate}

\subsubsection{Placeholder Gambar}
\begin{figure}[H]
    \centering
    \includegraphics[width=0.9\textwidth]{placeholder_arsip_dokumen.png}
    \caption{Halaman Arsip Dokumen - Tabel arsip dengan filter dan tombol download}
    \label{fig:arsip_dokumen}
\end{figure}

\newpage

\subsection{Performa GCG}

\subsubsection{Deskripsi}
Menu ini menampilkan analisis performa GCG dalam bentuk grafik dan statistik.

\subsubsection{Fitur Utama}
\begin{itemize}
    \item \textbf{Grafik Performa}: Visualisasi performa GCG
    \item \textbf{Statistik Detail}: Data statistik yang mendalam
    \item \textbf{Analisis Trend}: Analisis perkembangan performa
\end{itemize}

\subsubsection{Langkah Penggunaan}
\begin{enumerate}
    \item \textbf{Klik menu "Performa GCG"}
    \item \textbf{Pilih periode} yang ingin dianalisis
    \item \textbf{Lihat grafik} dan statistik yang ditampilkan
    \item \textbf{Analisis trend} dan performa
\end{enumerate}

\subsubsection{Placeholder Gambar}
\begin{figure}[H]
    \centering
    \includegraphics[width=0.9\textwidth]{placeholder_performa_gcg.png}
    \caption{Halaman Performa GCG - Grafik performa dan statistik detail}
    \label{fig:performa_gcg}
\end{figure}

\newpage

\subsection{AOI Management}

\subsubsection{Deskripsi}
Menu ini berfungsi untuk mengelola Area of Improvement (AOI) dan rekomendasi perbaikan.

\subsubsection{Fitur Utama}
\begin{itemize}
    \item \textbf{Manajemen AOI}: Membuat dan mengelola tabel AOI
    \item \textbf{Rekomendasi}: Menambah dan mengelola rekomendasi perbaikan
    \item \textbf{Tracking}: Melacak status implementasi rekomendasi
    \item \textbf{Prioritas}: Mengatur tingkat prioritas rekomendasi
\end{itemize}

\subsubsection{Langkah Penggunaan}

\paragraph{A. Manajemen Tabel AOI}
\begin{enumerate}
    \item \textbf{Klik menu "AOI Management"}
    \item \textbf{Pilih tahun buku} dari dropdown
    \item \textbf{Untuk menambah tabel AOI baru}:
    \begin{itemize}
        \item Klik "Tambah Tabel AOI"
        \item Isi deskripsi tabel
        \item Pilih target direktorat, subdirektorat, dan divisi
        \item Klik "Simpan"
    \end{itemize}
    \item \textbf{Untuk mengedit tabel}:
    \begin{itemize}
        \item Klik ikon edit pada tabel yang ingin diubah
        \item Ubah informasi sesuai kebutuhan
        \item Klik "Simpan"
    \end{itemize}
    \item \textbf{Untuk menghapus tabel}:
    \begin{itemize}
        \item Klik ikon hapus pada tabel yang ingin dihapus
        \item Konfirmasi penghapusan
    \end{itemize}
\end{enumerate}

\paragraph{B. Manajemen Rekomendasi}
\begin{enumerate}
    \item \textbf{Klik pada tabel AOI} yang ingin dikelola
    \item \textbf{Untuk menambah rekomendasi}:
    \begin{itemize}
        \item Klik "Tambah Rekomendasi"
        \item Isi detail rekomendasi:
        \begin{itemize}
            \item Nomor urut
            \item Jenis (Rekomendasi/Saran)
            \item Deskripsi rekomendasi
            \item Pihak terkait
            \item Tingkat urgensi (1-5)
            \item Jangka waktu
        \end{itemize}
        \item Klik "Simpan"
    \end{itemize}
    \item \textbf{Untuk mengedit rekomendasi}:
    \begin{itemize}
        \item Klik ikon edit pada rekomendasi yang ingin diubah
        \item Ubah informasi sesuai kebutuhan
        \item Klik "Simpan"
    \end{itemize}
    \item \textbf{Untuk menghapus rekomendasi}:
    \begin{itemize}
        \item Klik ikon hapus pada rekomendasi yang ingin dihapus
        \item Konfirmasi penghapusan
    \end{itemize}
\end{enumerate}

\subsubsection{Placeholder Gambar}
\begin{figure}[H]
    \centering
    \includegraphics[width=0.9\textwidth]{placeholder_aoi_management.png}
    \caption{Halaman AOI Management - Tabel AOI dengan tombol tambah, edit, dan hapus}
    \label{fig:aoi_management}
\end{figure}

\begin{figure}[H]
    \centering
    \includegraphics[width=0.7\textwidth]{placeholder_dialog_rekomendasi.png}
    \caption{Dialog Tambah Rekomendasi - Form rekomendasi dengan field lengkap}
    \label{fig:dialog_rekomendasi}
\end{figure}

\newpage

% Admin Menu
\section{Menu Admin}

\subsection{Dashboard Admin}

\subsubsection{Deskripsi}
Dashboard Admin menampilkan informasi dan statistik dokumen sesuai dengan subdirektorat yang ditugaskan.

\subsubsection{Fitur Utama}
\begin{itemize}
    \item \textbf{Statistik Pribadi}: Statistik dokumen yang diupload oleh admin
    \item \textbf{Daftar Dokumen}: Daftar dokumen yang telah diupload
    \item \textbf{Upload Dokumen}: Fitur untuk mengupload dokumen baru
    \item \textbf{Progress Tracking}: Melacak progress upload dokumen
\end{itemize}

\subsubsection{Langkah Penggunaan}
\begin{enumerate}
    \item \textbf{Klik menu "Dashboard"} (akan otomatis ke Dashboard Admin)
    \item \textbf{Pilih tahun buku} dari dropdown tahun
    \item \textbf{Lihat statistik} dokumen pribadi
    \item \textbf{Upload dokumen baru} dengan mengklik tombol "Upload Dokumen"
    \item \textbf{Lihat daftar dokumen} yang telah diupload
    \item \textbf{Download dokumen} jika diperlukan
\end{enumerate}

\subsubsection{Placeholder Gambar}
\begin{figure}[H]
    \centering
    \includegraphics[width=0.9\textwidth]{placeholder_dashboard_admin.png}
    \caption{Dashboard Admin - Statistik pribadi, tombol upload, dan daftar dokumen}
    \label{fig:dashboard_admin}
\end{figure}

\begin{figure}[H]
    \centering
    \includegraphics[width=0.7\textwidth]{placeholder_dialog_upload_admin.png}
    \caption{Dialog Upload Dokumen Admin - Form upload dengan pilihan checklist}
    \label{fig:dialog_upload_admin}
\end{figure}

\newpage

% General Features
\section{Fitur Umum}

\subsection{Manajemen Profil}

\subsubsection{Langkah Penggunaan}
\begin{enumerate}
    \item \textbf{Klik nama pengguna} di pojok kanan atas
    \item \textbf{Pilih "Profil"} dari dropdown menu
    \item \textbf{Edit informasi} yang diperlukan
    \item \textbf{Klik "Simpan"} untuk menyimpan perubahan
\end{enumerate}

\subsubsection{Placeholder Gambar}
\begin{figure}[H]
    \centering
    \includegraphics[width=0.6\textwidth]{placeholder_menu_profil.png}
    \caption{Menu Profil - Dropdown menu dengan opsi profil, pengaturan, dan logout}
    \label{fig:menu_profil}
\end{figure}

\subsection{Logout}

\subsubsection{Langkah Penggunaan}
\begin{enumerate}
    \item \textbf{Klik nama pengguna} di pojok kanan atas
    \item \textbf{Pilih "Logout"} dari dropdown menu
    \item \textbf{Konfirmasi logout}
    \item \textbf{Sistem akan mengarahkan} ke halaman login
\end{enumerate}

\subsection{Responsive Design}

\subsubsection{Fitur Mobile}
\begin{itemize}
    \item \textbf{Sidebar Collapsible}: Sidebar dapat di-collapse di layar kecil
    \item \textbf{Touch Friendly}: Tombol dan menu yang mudah disentuh
    \item \textbf{Adaptive Layout}: Layout yang menyesuaikan ukuran layar
\end{itemize}

\subsubsection{Fitur Desktop}
\begin{itemize}
    \item \textbf{Full Sidebar}: Sidebar selalu terbuka di layar besar
    \item \textbf{Hover Effects}: Efek hover pada tombol dan menu
    \item \textbf{Keyboard Navigation}: Navigasi menggunakan keyboard
\end{itemize}

\newpage

% Troubleshooting
\section{Troubleshooting}

\subsection{Masalah Umum dan Solusi}

\subsubsection{Tidak Bisa Login}
\textbf{Gejala}: Error saat memasukkan username/password\\
\textbf{Solusi}:
\begin{itemize}
    \item Pastikan username dan password benar
    \item Periksa koneksi internet
    \item Clear cache browser
    \item Hubungi administrator sistem
\end{itemize}

\subsubsection{File Upload Gagal}
\textbf{Gejala}: File tidak bisa diupload\\
\textbf{Solusi}:
\begin{itemize}
    \item Periksa ukuran file (maksimal 10MB)
    \item Pastikan format file didukung (PDF, DOC, DOCX)
    \item Periksa koneksi internet
    \item Coba upload file lain
\end{itemize}

\subsubsection{Halaman Tidak Muncul}
\textbf{Gejala}: Halaman kosong atau error\\
\textbf{Solusi}:
\begin{itemize}
    \item Refresh halaman (F5)
    \item Clear cache browser
    \item Periksa koneksi internet
    \item Logout dan login kembali
\end{itemize}

\subsubsection{Data Tidak Terupdate}
\textbf{Gejala}: Perubahan data tidak muncul\\
\textbf{Solusi}:
\begin{itemize}
    \item Refresh halaman
    \item Periksa tahun buku yang dipilih
    \item Logout dan login kembali
    \item Hubungi administrator sistem
\end{itemize}

\subsection{Kontak Support}
Jika mengalami masalah yang tidak dapat diselesaikan:
\begin{itemize}
    \item \textbf{Email}: support@gcg-system.com
    \item \textbf{Telepon}: (021) 1234-5678
    \item \textbf{WhatsApp}: +62 812-3456-7890
\end{itemize}

\newpage

% Conclusion
\section{Kesimpulan}

Manual book ini telah menjelaskan secara lengkap cara menggunakan aplikasi GCG Documents Management System. Pengguna diharapkan dapat:

\begin{enumerate}
    \item \textbf{Memahami struktur} aplikasi dan hierarki pengguna
    \item \textbf{Menggunakan fitur} sesuai dengan role yang dimiliki
    \item \textbf{Mengelola dokumen} GCG secara efektif
    \item \textbf{Memecahkan masalah} umum yang mungkin terjadi
\end{enumerate}

\subsection{Tips Penggunaan}
\begin{itemize}
    \item \textbf{Baca manual} sebelum menggunakan aplikasi
    \item \textbf{Simpan dokumen} secara teratur
    \item \textbf{Backup data} penting secara berkala
    \item \textbf{Update password} secara berkala
    \item \textbf{Logout} setelah selesai menggunakan aplikasi
\end{itemize}

\subsection{Update Manual}
Manual ini akan diupdate secara berkala sesuai dengan perkembangan aplikasi. Pastikan selalu menggunakan versi manual yang terbaru.

% Appendix
\appendix
\section{Glossary}
\begin{description}
    \item[GCG] Good Corporate Governance - Tata kelola perusahaan yang baik
    \item[AOI] Area of Improvement - Area yang perlu diperbaiki
    \item[Admin] Pengguna dengan hak akses terbatas
    \item[Super Admin] Pengguna dengan hak akses penuh
    \item[User] Pengguna dengan hak akses dasar
\end{description}

\end{document}

